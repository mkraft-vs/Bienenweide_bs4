% Options for packages loaded elsewhere
\PassOptionsToPackage{unicode}{hyperref}
\PassOptionsToPackage{hyphens}{url}
%
\documentclass[
]{book}
\usepackage{amsmath,amssymb}
\usepackage{iftex}
\ifPDFTeX
  \usepackage[T1]{fontenc}
  \usepackage[utf8]{inputenc}
  \usepackage{textcomp} % provide euro and other symbols
\else % if luatex or xetex
  \usepackage{unicode-math} % this also loads fontspec
  \defaultfontfeatures{Scale=MatchLowercase}
  \defaultfontfeatures[\rmfamily]{Ligatures=TeX,Scale=1}
\fi
\usepackage{lmodern}
\ifPDFTeX\else
  % xetex/luatex font selection
\fi
% Use upquote if available, for straight quotes in verbatim environments
\IfFileExists{upquote.sty}{\usepackage{upquote}}{}
\IfFileExists{microtype.sty}{% use microtype if available
  \usepackage[]{microtype}
  \UseMicrotypeSet[protrusion]{basicmath} % disable protrusion for tt fonts
}{}
\makeatletter
\@ifundefined{KOMAClassName}{% if non-KOMA class
  \IfFileExists{parskip.sty}{%
    \usepackage{parskip}
  }{% else
    \setlength{\parindent}{0pt}
    \setlength{\parskip}{6pt plus 2pt minus 1pt}}
}{% if KOMA class
  \KOMAoptions{parskip=half}}
\makeatother
\usepackage{xcolor}
\usepackage{longtable,booktabs,array}
\usepackage{calc} % for calculating minipage widths
% Correct order of tables after \paragraph or \subparagraph
\usepackage{etoolbox}
\makeatletter
\patchcmd\longtable{\par}{\if@noskipsec\mbox{}\fi\par}{}{}
\makeatother
% Allow footnotes in longtable head/foot
\IfFileExists{footnotehyper.sty}{\usepackage{footnotehyper}}{\usepackage{footnote}}
\makesavenoteenv{longtable}
\usepackage{graphicx}
\makeatletter
\def\maxwidth{\ifdim\Gin@nat@width>\linewidth\linewidth\else\Gin@nat@width\fi}
\def\maxheight{\ifdim\Gin@nat@height>\textheight\textheight\else\Gin@nat@height\fi}
\makeatother
% Scale images if necessary, so that they will not overflow the page
% margins by default, and it is still possible to overwrite the defaults
% using explicit options in \includegraphics[width, height, ...]{}
\setkeys{Gin}{width=\maxwidth,height=\maxheight,keepaspectratio}
% Set default figure placement to htbp
\makeatletter
\def\fps@figure{htbp}
\makeatother
\setlength{\emergencystretch}{3em} % prevent overfull lines
\providecommand{\tightlist}{%
  \setlength{\itemsep}{0pt}\setlength{\parskip}{0pt}}
\setcounter{secnumdepth}{5}
\usepackage{booktabs}
\ifLuaTeX
  \usepackage{selnolig}  % disable illegal ligatures
\fi
\usepackage[]{natbib}
\bibliographystyle{apalike}
\IfFileExists{bookmark.sty}{\usepackage{bookmark}}{\usepackage{hyperref}}
\IfFileExists{xurl.sty}{\usepackage{xurl}}{} % add URL line breaks if available
\urlstyle{same}
\hypersetup{
  pdftitle={Bienenweide},
  pdfauthor={Manfred Kraft et al.},
  hidelinks,
  pdfcreator={LaTeX via pandoc}}

\title{Bienenweide}
\author{Manfred Kraft et al.}
\date{2023-08-08}

\begin{document}
\maketitle

{
\setcounter{tocdepth}{1}
\tableofcontents
}
\hypertarget{about}{%
\chapter{About}\label{about}}

Bemerkungen zur Rolle des Obmann's für Bienenweide

\hypertarget{einfuxfchrung}{%
\chapter{Einführung}\label{einfuxfchrung}}

\hypertarget{biodiversituxe4t}{%
\section{Biodiversität}\label{biodiversituxe4t}}

\begin{itemize}
\tightlist
\item
  Biodiversität in ihrer Gesamtheit
\item
  Definition des Begriffs
\item
  Konsequenzen innerhalb der vernetzten Natur
\item
  Dynamik durch Umweltveränderung-
\item
  Anpassung durch Evolution
\item
  Vielfalt in allen Lebensräumen
\item
  Allgemeindarstellung der Lebensräume und Nistplätze
\end{itemize}

\hypertarget{entwicklung-der-biodiversituxe4t}{%
\section{Entwicklung der Biodiversität}\label{entwicklung-der-biodiversituxe4t}}

\hypertarget{entwicklung-pflanzen-insbesondere-bluxfctenpflanzen}{%
\subsection{Entwicklung Pflanzen, insbesondere Blütenpflanzen}\label{entwicklung-pflanzen-insbesondere-bluxfctenpflanzen}}

\hypertarget{entwicklung-insekten-insbesondere-bestuxe4ubungsinsekten}{%
\subsection{Entwicklung Insekten, insbesondere Bestäubungsinsekten}\label{entwicklung-insekten-insbesondere-bestuxe4ubungsinsekten}}

\hypertarget{entwicklung-der-davon-abhuxe4ngigen-kreaturenarten}{%
\subsection{Entwicklung der davon abhängigen Kreaturen/Arten}\label{entwicklung-der-davon-abhuxe4ngigen-kreaturenarten}}

\hypertarget{konsequenzen-des-biodiversituxe4ts-verlustes}{%
\section{Konsequenzen des Biodiversitäts-Verlustes}\label{konsequenzen-des-biodiversituxe4ts-verlustes}}

\hypertarget{wiederherstellung-aller-verfuxfcgbaren-und-muxf6glichen-lebensruxe4ume}{%
\subsection{Wiederherstellung aller verfügbaren und möglichen Lebensräume}\label{wiederherstellung-aller-verfuxfcgbaren-und-muxf6glichen-lebensruxe4ume}}

\hypertarget{unter-einbeziehung-aller-verwaltung-buxfcrger-und-gewerbe}{%
\subsection{Unter Einbeziehung aller: Verwaltung, Bürger und Gewerbe}\label{unter-einbeziehung-aller-verwaltung-buxfcrger-und-gewerbe}}

\hypertarget{landwirtschaft}{%
\subsection{Landwirtschaft}\label{landwirtschaft}}

\hypertarget{moderator-funktion-obmannfrau-fuxfcr-bienenweide}{%
\subsection{Moderator-Funktion: Obmann/Frau für Bienenweide}\label{moderator-funktion-obmannfrau-fuxfcr-bienenweide}}

\hypertarget{aufgaben-des-obmanns-fuxfcr-bienenweide}{%
\subsection{Aufgaben des Obmanns für Bienenweide}\label{aufgaben-des-obmanns-fuxfcr-bienenweide}}

\hypertarget{honigbienen-und-ihre-verwandten}{%
\chapter{Honigbienen und ihre Verwandten}\label{honigbienen-und-ihre-verwandten}}

\hypertarget{gemeinsamkeiten-der-bienen}{%
\section{Gemeinsamkeiten der Bienen}\label{gemeinsamkeiten-der-bienen}}

\hypertarget{entwicklungsgeschichte}{%
\subsection{Entwicklungsgeschichte}\label{entwicklungsgeschichte}}

\hypertarget{kuxf6rperbau}{%
\subsection{Körperbau}\label{kuxf6rperbau}}

\hypertarget{vermehrung}{%
\subsection{Vermehrung}\label{vermehrung}}

\hypertarget{wildbienen}{%
\subsubsection{Wildbienen}\label{wildbienen}}

\hypertarget{honigbienen}{%
\subsubsection{Honigbienen}\label{honigbienen}}

\hypertarget{nahrung-allgemein}{%
\section{Nahrung allgemein}\label{nahrung-allgemein}}

\hypertarget{nektar-und-nektarien}{%
\subsection{Nektar und Nektarien}\label{nektar-und-nektarien}}

\hypertarget{pollen}{%
\subsection{Pollen}\label{pollen}}

\hypertarget{waldtracht}{%
\subsection{Waldtracht}\label{waldtracht}}

\hypertarget{sonstige-uxf6l}{%
\subsection{Sonstige (Öl)}\label{sonstige-uxf6l}}

\hypertarget{oligolekten-und-generalisten}{%
\section{Oligolekten und Generalisten}\label{oligolekten-und-generalisten}}

\hypertarget{nahrungspflanzen}{%
\subsection{Nahrungspflanzen}\label{nahrungspflanzen}}

\hypertarget{spezialisten}{%
\subsubsection{Spezialisten}\label{spezialisten}}

\hypertarget{generalisten}{%
\subsubsection{Generalisten}\label{generalisten}}

\hypertarget{konkurrenz-honigbienen-vs.-wildbiene}{%
\subsection{Konkurrenz Honigbienen vs.~Wildbiene}\label{konkurrenz-honigbienen-vs.-wildbiene}}

\hypertarget{konkurrenz-wildbiene-vs.-gezuxfcchtete-wildbiene}{%
\subsection{Konkurrenz Wildbiene vs.~gezüchtete Wildbiene}\label{konkurrenz-wildbiene-vs.-gezuxfcchtete-wildbiene}}

\hypertarget{landwirtschaft-und-die-imker}{%
\chapter{Landwirtschaft und die Imker}\label{landwirtschaft-und-die-imker}}

\hypertarget{historische-betrachtung}{%
\section{Historische Betrachtung}\label{historische-betrachtung}}

\hypertarget{phazelia-und-andere-kulturpflanzen-wie-raps-etc.}{%
\section{Phazelia und Andere Kulturpflanzen wie Raps, etc.}\label{phazelia-und-andere-kulturpflanzen-wie-raps-etc.}}

Mehrertrag für die Landwirtschaft

\hypertarget{biogas}{%
\section{Biogas}\label{biogas}}

\hypertarget{bienenweide}{%
\section{Bienenweide}\label{bienenweide}}

\hypertarget{bienenweide-1}{%
\chapter{Bienenweide}\label{bienenweide-1}}

\hypertarget{trachtflieuxdfband}{%
\section{Trachtfließband}\label{trachtflieuxdfband}}

\hypertarget{bluxfchwiesen}{%
\section{Blühwiesen}\label{bluxfchwiesen}}

\hypertarget{trachten}{%
\section{Trachten}\label{trachten}}

\hypertarget{heuwiesen}{%
\subsection{Heuwiesen}\label{heuwiesen}}

\hypertarget{obstbuxe4ume}{%
\subsection{Obstbäume}\label{obstbuxe4ume}}

\hypertarget{suxe4ume}{%
\subsection{Säume}\label{suxe4ume}}

\hypertarget{struxe4ucher}{%
\subsection{Sträucher}\label{struxe4ucher}}

\hypertarget{buxe4ume}{%
\subsection{Bäume}\label{buxe4ume}}

\hypertarget{nadelbuxe4ume}{%
\subsubsection{Nadelbäume}\label{nadelbuxe4ume}}

\hypertarget{laubbuxe4ume}{%
\subsubsection{Laubbäume}\label{laubbuxe4ume}}

\hypertarget{saatgut}{%
\section{Saatgut}\label{saatgut}}

\hypertarget{rechtsvorschriften-40-umschg}{%
\subsection{Rechtsvorschriften, \$40 UmSchG}\label{rechtsvorschriften-40-umschg}}

\hypertarget{regionales-saatgut}{%
\subsection{Regionales Saatgut}\label{regionales-saatgut}}

\hypertarget{bluxfchfluxe4chen}{%
\subsubsection{Blühflächen}\label{bluxfchfluxe4chen}}

\hypertarget{suxe4ume-1}{%
\subsubsection{Säume}\label{suxe4ume-1}}

\hypertarget{wiesendrusch}{%
\subsection{Wiesendrusch}\label{wiesendrusch}}

\hypertarget{anlage-und-pflege-der-bluxfchwiesen}{%
\section{Anlage und Pflege der Blühwiesen}\label{anlage-und-pflege-der-bluxfchwiesen}}

\hypertarget{saatgut-auswahl}{%
\subsection{Saatgut-Auswahl}\label{saatgut-auswahl}}

\hypertarget{fluxe4che-vorbereiten}{%
\subsection{Fläche vorbereiten}\label{fluxe4che-vorbereiten}}

\hypertarget{einsaat}{%
\subsection{Einsaat}\label{einsaat}}

\hypertarget{pflege}{%
\subsection{Pflege}\label{pflege}}

\hypertarget{weitere-unterstuxfctzung-der-wildbienen}{%
\section{Weitere Unterstützung der Wildbienen}\label{weitere-unterstuxfctzung-der-wildbienen}}

\hypertarget{sandarium}{%
\subsection{Sandarium}\label{sandarium}}

\hypertarget{nisthabitate}{%
\subsection{Nisthabitate}\label{nisthabitate}}

\hypertarget{bodenbruxfcter}{%
\subsubsection{Bodenbrüter}\label{bodenbruxfcter}}

\hypertarget{stuxe4ngelbruxfcter}{%
\subsubsection{Stängelbrüter}\label{stuxe4ngelbruxfcter}}

\hypertarget{steilwandbruxfcter}{%
\subsubsection{Steilwandbrüter}\label{steilwandbruxfcter}}

\hypertarget{schneckenhausbruxfcter}{%
\subsubsection{Schneckenhausbrüter}\label{schneckenhausbruxfcter}}

\hypertarget{waagrechtbruxfcter}{%
\subsubsection{Waagrechtbrüter}\label{waagrechtbruxfcter}}

\hypertarget{huxf6hlenbruxfcter}{%
\subsubsection{Höhlenbrüter}\label{huxf6hlenbruxfcter}}

\hypertarget{trittstein-konzept}{%
\section{Trittstein-Konzept}\label{trittstein-konzept}}

\hypertarget{pflanzen}{%
\chapter{Pflanzen}\label{pflanzen}}

\hypertarget{trachtwerte-der-bluxfchpflanzen}{%
\section{Trachtwerte der Blühpflanzen}\label{trachtwerte-der-bluxfchpflanzen}}

Liste der wichtigsten Blühpflanzen mit ihren Trachtwerten

\hypertarget{stauden-struxe4ucher-und-buxe4ume}{%
\section{Stauden, Sträucher und Bäume}\label{stauden-struxe4ucher-und-buxe4ume}}

\hypertarget{suxe4ume-2}{%
\subsection{Säume}\label{suxe4ume-2}}

\hypertarget{wichtige-saumpflanzen}{%
\subsubsection{Wichtige Saumpflanzen}\label{wichtige-saumpflanzen}}

\hypertarget{anlegen-von-suxe4umen}{%
\subsubsection{Anlegen von Säumen}\label{anlegen-von-suxe4umen}}

\hypertarget{struxe4ucher-1}{%
\subsection{Sträucher}\label{struxe4ucher-1}}

Wichtige Sträucher

\hypertarget{buxe4ume-fuxfcr-pollen-und-nektar}{%
\subsection{Bäume für Pollen und Nektar}\label{buxe4ume-fuxfcr-pollen-und-nektar}}

Wichtige Bäume

\hypertarget{trachtwerte}{%
\section{Trachtwerte}\label{trachtwerte}}

Liste der wichtigsten Pflanzen mit ihren Trachtwerten

\hypertarget{oligolekten-und-deren-pflanzen}{%
\section{Oligolekten und deren Pflanzen}\label{oligolekten-und-deren-pflanzen}}

Liste der wichtigsten Pflanzen für spezialisierte Wildbienen

\hypertarget{uxf6ffentlichkeitsarbeit}{%
\chapter{Öffentlichkeitsarbeit}\label{uxf6ffentlichkeitsarbeit}}

\hypertarget{imker-und-biodiverstuxe4t}{%
\section{Imker und Biodiverstät}\label{imker-und-biodiverstuxe4t}}

\hypertarget{kontrolleure-bei-gift-in-der-landwirtschaft-u.-uxe4.}{%
\subsection{Kontrolleure bei Gift in der Landwirtschaft u. ä.}\label{kontrolleure-bei-gift-in-der-landwirtschaft-u.-uxe4.}}

\hypertarget{verbesserung-des-bluxfch-angebots-fuxfcr-alle-bestuxe4uber}{%
\subsection{Verbesserung des Blüh-Angebots für alle Bestäuber}\label{verbesserung-des-bluxfch-angebots-fuxfcr-alle-bestuxe4uber}}

\hypertarget{fluxe4chenwirsamkeit-mitgliedersind-in-der-fluxe4che-aktiv}{%
\subsection{Flächenwirsamkeit: Mitgliedersind in der Fläche aktiv}\label{fluxe4chenwirsamkeit-mitgliedersind-in-der-fluxe4che-aktiv}}

\hypertarget{footnotes-and-citations}{%
\chapter{Footnotes and citations}\label{footnotes-and-citations}}

\hypertarget{footnotes}{%
\section{Footnotes}\label{footnotes}}

Footnotes are put inside the square brackets after a caret \texttt{\^{}{[}{]}}. Like this one \footnote{This is a footnote.}.

\hypertarget{citations}{%
\section{Citations}\label{citations}}

Reference items in your bibliography file(s) using \texttt{@key}.

For example, we are using the \textbf{bookdown} package \citep{R-bookdown} (check out the last code chunk in index.Rmd to see how this citation key was added) in this sample book, which was built on top of R Markdown and \textbf{knitr} \citep{xie2015} (this citation was added manually in an external file book.bib).
Note that the \texttt{.bib} files need to be listed in the index.Rmd with the YAML \texttt{bibliography} key.

The RStudio Visual Markdown Editor can also make it easier to insert citations: \url{https://rstudio.github.io/visual-markdown-editing/\#/citations}

  \bibliography{book.bib,packages.bib}

\end{document}
